%
%  This is ``front matter'' for the thesis.
%
%  Regarding ``References'' below:
%      KEY    MEANING
%      PU     ``A Manual for the Preparation of Graduate Theses'',
%             The Graduate School, Purdue University, 1996.
%      TCMOS  The Chicago Manual of Style, Edition 14.
%      WNNCD  Webster's Ninth New Collegiate Dictionary.
%
%  Lines marked with "%%" may need to be changed.
%

  % Dedication page is optional.
  % A name and often a message in tribute to a person or cause.
  % References: PU 15, WNNCD 332.
\begin{dedication}
  This is the dedication.
\end{dedication}

  % Acknowledgements page is optional but most theses include
  % a brief statement of appreciation or recognition of special
  % assistance.
  % Reference: PU 16.
\begin{acknowledgments}
  I would like to thank my advisor and my committee.
\end{acknowledgments}

  % The preface is optional.
  % References: PU 16, TCMOS 1.49, WNNCD 927.
% \begin{preface}
%   This is the preface.
% \end{preface}

  % The Table of Contents is required.
  % The Table of Contents will be automatically created for you
  % using information you supply in
  %     \chapter
  %     \section
  %     \subsection
  %     \subsubsection
  % commands.
  % Reference: PU 16.
\tableofcontents

  % If your thesis has tables, a list of tables is required.
  % The List of Tables will be automatically created for you using
  % information you supply in
  %     \begin{table} ... \end{table}
  % environments.
  % Reference: PU 16.
\listoftables

  % If your thesis has figures, a list of figures is required.
  % The List of Figures will be automatically created for you using
  % information you supply in
  %     \begin{figure} ... \end{figure}
  % environments.
  % Reference: PU 16.
\listoffigures

  % List of Symbols is optional.
  % Reference: PU 17.
% \begin{symbols}
%   $m$& mass\cr
%   $v$& velocity\cr
% \end{symbols}

  % List of Abbreviations is optional.
  % Reference: PU 17.
% \begin{abbreviations}
%   abbr& abbreviation\cr
%   bcf& billion cubic feet\cr
%   BMOC& big man on campus\cr
% \end{abbreviations}

  % Nomenclature is optional.
  % Reference: PU 17.
% \begin{nomenclature}
%   Alanine& 2-Aminopropanoic acid\cr
%   Valine& 2-Amino-3-methylbutanoic acid\cr
% \end{nomenclature}

  % Glossary is optional
  % Reference: PU 17.
% \begin{glossary}
%   chick& female, usually young\cr
%   dude& male, usually young\cr
% \end{glossary}

  % Abstract is required.
\begin{abstract}
    Cell chemotaxis is crucial to many biological functions. Chemotaxis is the process in which cell migrate in response to chemical concentration gradients detected in the environment. Recent experiments show that cells are capable of detecting shallow gradients that are on the order of 10 molecule difference across a cell body. This effect is heightened when examining collectives of cells. Examples from morphogenesis and cancer metastasis demonstrate that cell collectives respond to gradients equivalent to a 1 molecule difference in concentration across a cell body. While the physical constraints to cell gradient sensing are well understood, how the sensory information leads to cell migration, and coherent multicellular movement in the case of collectives, remains poorly understood.
    Here we examine how sensory error leads to error in cell chemotactic performance. First, we study single cell chemotaxis and use both simulations and analytical models to place physical constraints on chemotactic performance. Next we turn our attention to collective chemotaxis. In development, wound healing, and cancer metastasis among others, cells migrate collectively. We examine how collective cell interactions can possibly improve chemotactic performance. We develop a novel model for quantifying the physical limit to chemotactic precision for two stereotypical modes of collective chemotaxis. Finally, we conclude by examining the effects of intercellular communication on collective chemotaxis. We use simulations to test how well collectives can chemotax through very shallow gradients with the help of communication.
    \red{ The importance of this work is to address the gap in knowledge between the physical constraints on cell sensing, and the modeling of chemotaxis. Since experiments show that individual cells and collectives are capable of chemotaxis in response to very small chemical signals it is very important to understand how sensory noise affects the ability to chemotax.}

    % Collective cell migration in response to a chemical cue requires both multicellular sensing of chemical gradients and coordinated mechanical action. Examples from morphogenesis and cancer metastasis demonstrate that clusters of migratory cells are extremely sensitive, responding to gradients of less than 1\% difference in chemical concentration across a cell body. While the limits to multicellular sensing are becoming known, how this information leads to coherent migration remains poorly understood. We develop a model of multicellular sensing and migration in which groups of cells collectively measure noisy chemical gradients. The output of the sensing process is coupled to individual cells’ polarization to model migratory behavior. Through the use of numerical simulations, we find that larger clusters of cells detect the gradient direction with higher precision and thus achieve stronger polarization bias, but larger clusters are also accompanied by less coherent collective motion.
    % The trade-off between these two effects leads to an optimally efficient cluster size. Experimental tests of our model are ongoing and are focused on observations of breast cancer cell migration. Future plans include extending the model to systems of cells which include leading-edge cells that are phenotypically different from non-leading cells, and exploring an alternative model of collective migration in which cells make independent measurements and coordinated behavior emerges through local interactions. By completing these studies we aim to understand the precise roles of multicellular sensing in producing collective cell migration, in metastasis and in general.
\end{abstract}
