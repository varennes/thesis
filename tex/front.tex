%
%  This is ``front matter'' for the thesis.
%
%  Regarding ``References'' below:
%      KEY    MEANING
%      PU     ``A Manual for the Preparation of Graduate Theses'',
%             The Graduate School, Purdue University, 1996.
%      TCMOS  The Chicago Manual of Style, Edition 14.
%      WNNCD  Webster's Ninth New Collegiate Dictionary.
%
%  Lines marked with "%%" may need to be changed.
%

  % Dedication page is optional.
  % A name and often a message in tribute to a person or cause.
  % References: PU 15, WNNCD 332.
% \begin{dedication}
%   This is the dedication.
% \end{dedication}

  % Acknowledgements page is optional but most theses include
  % a brief statement of appreciation or recognition of special
  % assistance.
  % Reference: PU 16.
\begin{acknowledgments}
  First, I would like to thank my adviser Andrew Mugler, for mentoring me through my graduate career. Andrew has been a wonderful adviser, I could not have done this work without his helpful, positive and welcoming stewardship over the years. I would also like to thank Bumsoo Han, Paul Muzikar, and Ken Ritchie of my thesis committee for their guidance and countless helpful conversations over the years.

  My fellow research group member Sean Fancher has been an invaluable research and really helped me in developing some of the mathematical framework found here. I would especially like to thank Hye-ran Moon and Bumsoo Han for their contributions to our collaborative project of cancer cell migration. It has been a great experience to be part of an interdisciplinary team working on such a cool project. Additionally, having had Bumsoo's perspective and feedback on my ongoing theoretical and computational projects has really helped improve their value and significance. I also want to thank Shivam Gupta for being a great coworker.

  I want to thank the members of the Mugler research group: Tommy Byrd, Sean Fancher, Xiaoling Zhai, Mike Vennettilli, and Andrew Mugler for their continual support of my work.

  I want to thank Sandy Formica, Linda Paquay, and Janice Thomaz for their helping me through the logisitcs and bureaucracy of graduate school.

  Finally, I would sincerely like to thank my family and friends for their support. My partner, Elli Olson for her continued moral support and 
\end{acknowledgments}

  % The preface is optional.
  % References: PU 16, TCMOS 1.49, WNNCD 927.
% \begin{preface}
%   This is the preface.
% \end{preface}

  % The Table of Contents is required.
  % The Table of Contents will be automatically created for you
  % using information you supply in
  %     \chapter
  %     \section
  %     \subsection
  %     \subsubsection
  % commands.
  % Reference: PU 16.
\tableofcontents

  % If your thesis has tables, a list of tables is required.
  % The List of Tables will be automatically created for you using
  % information you supply in
  %     \begin{table} ... \end{table}
  % environments.
  % Reference: PU 16.
\listoftables

  % If your thesis has figures, a list of figures is required.
  % The List of Figures will be automatically created for you using
  % information you supply in
  %     \begin{figure} ... \end{figure}
  % environments.
  % Reference: PU 16.
\listoffigures

  % List of Symbols is optional.
  % Reference: PU 17.
% \begin{symbols}
%   $m$& mass\cr
%   $v$& velocity\cr
% \end{symbols}

  % List of Abbreviations is optional.
  % Reference: PU 17.
% \begin{abbreviations}
%   abbr& abbreviation\cr
%   bcf& billion cubic feet\cr
%   BMOC& big man on campus\cr
% \end{abbreviations}

  % Nomenclature is optional.
  % Reference: PU 17.
% \begin{nomenclature}
%   Alanine& 2-Aminopropanoic acid\cr
%   Valine& 2-Amino-3-methylbutanoic acid\cr
% \end{nomenclature}

  % Glossary is optional
  % Reference: PU 17.
% \begin{glossary}
%   chick& female, usually young\cr
%   dude& male, usually young\cr
% \end{glossary}

  % Abstract is required.
\begin{abstract}
    Cell chemotaxis is crucial to many biological functions including development, wound healing, and cancer metastasis. Chemotaxis is the process in which cells migrate in response to chemical concentration gradients detected in the environment. Recent experiments show that cells are capable of detecting shallow gradients that are on the order of a 10 molecule difference across a cell body.
    Detection can be improved by an additional order of magnitude when examining multicellular collectives.
    Examples from morphogenesis and cancer metastasis demonstrate collective response to gradients equivalent to a 1 molecule difference in concentration across a cell body. While the physical constraints to cell gradient sensing are well understood, how the sensory information leads to cell migration, and coherent multicellular movement in the case of collectives, remains poorly understood.
    Here we examine how extrinsic sensory noise leads to error in chemotactic performance. First, we study single cell chemotaxis and use both simulations and analytical models to place physical constraints on chemotactic performance. Next we turn our attention to collective chemotaxis. We examine how collective cell interactions can improve chemotactic performance. We develop a novel model for quantifying the physical limit to chemotactic precision for two stereotypical modes of collective chemotaxis. Finally, we conclude by examining the effects of intercellular communication on collective chemotaxis. We use simulations to test how well collectives can chemotax through very shallow gradients with the help of communication.
    By studying these computational and theoretical models of individual and collective chemotaxis, we address the gap in knowledge between chemical sensing and directed migration.
\end{abstract}
