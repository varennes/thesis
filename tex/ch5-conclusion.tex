
\chapter{Conclusion}

\noindent
\red{Copy-pasted abstract below}
Cell chemotaxis is crucial to many biological functions. Chemotaxis is the process in which cell migrate in response to chemical concentration gradients detected in the environment. Recent experiments show that cells are capable of detecting shallow gradients that are on the order of 10 molecule difference across a cell body. This effect is heightened when examining collectives of cells. Examples from morphogenesis and cancer metastasis demonstrate that cell collectives respond to gradients equivalent to a 1 molecule difference in concentration across a cell body. While the physical constraints to cell gradient sensing are well understood, how the sensory information leads to cell migration, and coherent multicellular movement in the case of collectives, remains poorly understood.
Here we examine how sensory error leads to error in cell chemotactic performance. First, we study single cell chemotaxis and model use both simulations and analytical models to place physical constraints on chemotactic performance. Next we turn our attention to collective chemotaxis. In development, wound healing, and cancer metastasis among others, cells migrate collectively. We examine how collective cell interactions may improve chemotactic performance. We develop a novel model for quantifying the physical limit to chemotactic precision for to stereotypical modes of collective chemotaxis. Finally, we conclude by examining the effects of intercellular communication on collective chemotaxis. We use simulations to test how well collectives can chemotax through very shallow gradients with the help of communication.
\red{Add statement of significance.}

\red{Add some sort of outlook, perspective.}
