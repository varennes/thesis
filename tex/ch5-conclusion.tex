
\chapter{Conclusion}

\noindent
Cell chemotaxis is crucial to many biological functions. As discussed in Ch.\ 1, it is critical to growth, nutrient search, development, wound healing, and in several instances, cancer metastasis. Chemotaxis can involve individual cells or collectives migrating in response to chemical concentration gradients. Recently, studies have shown the incredible precision of cell sensing. Detection of shallow gradients that are on the order of a 10 molecule difference across a cell body has been observed. Even more remarkable is that this precision is heightened in cell collectives. Examples from morphogenesis and cancer metastasis demonstrate that collectives can sense gradients an order of magnitude smaller than what's possible for single cells. Although the physical constraints to gradient sensing are well understood and reviewed in Ch.\ 1, how sensing leads to coherent, directed migration remains poorly understood. With this problem in mind, we set out to understand and quantify how the physical limits of chemical sensing lead to constraints on chemotactic performance.

We began by studying the individual chemotaxis of breast cancer cells in Ch.\ 2. In collaboration with Dr.\ Han's research group we used experiments, simulations, and analytical models to place physical constraints on the cells' chemotactic performance. From the simulations we identified the dependence of chemotaxis precision, persistence and speed on crucial environmental parameters like background concentration, gradient, and ECM stiffness. From our analytical approach we found that a biased persistent random walk places bounds on the precision and persistence of the breast cancer cells. In Ch.\ 3, we turned our attention to collective chemotaxis. We developed a novel analytical model that predicts the physical limits of chemotactic precision for two generic classes of collective migration. We found that collective dimensionality is crucial to understanding how correlations between sensory cells cascades to the noise in the collective's perceived gradient direction. Lastly, in Ch.\ 4 we studied an application of the EC class of migration from Ch.\ 3 where communication between cells is explicitly accounted for. Using simulations we test the chemotactic performance of cell collectives in gradients too shallow for single cell detection. Here we again find that chemotactic performance depends on the size of the collective, and it was also shown to depend on the efficacy of intercellular communication.

% Experiments have shown that individual cells as well as collectives are capable of chemotaxis in response to very small chemical signals. From studies on the physical limits to chemical signaling we know that these chemotaxing cells often behave in environments very near these limits, therefore it is important to understand how sensory noise cascades into chemotactic performance. Here in this work we present computational and analytical approaches to attacking these problems. They are important step towards developing a complete predictive model of cell chemotaxis.
